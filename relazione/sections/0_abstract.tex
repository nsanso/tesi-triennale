\chapter*{Abstract}
Nelle città di oggi sono sempre più diffusi sistemi di sorveglianza a camera fissa.
Tali sistemi consentono di monitorare il traffico cittadino, rendendo le città più sicure.
Grazie al progresso degli ultimi anni nei campi di Machine Learning e Object Detection, è possibile rilevare in tempo reale entità, come veicoli e persone, inquadrati da tali camere.
Risulta quindi di interesse estrarre informazioni accurate riguardo alla posizione e alla velocità di tali entità. 
Tali informazioni possono essere utilizzate in tempo reale nella fase di tracciamento delle entità, e nello sviluppo di euristiche per il rilevamento di anomalie.
Queste anomalie possono essere notificate alle autorità in modo da ridurre i tempi di risposta in presenza di situazioni di rischio.
Possono anche essere utilizzate in un secondo tempo per studiare e modellare il comportamento delle entità, così da sviluppare interventi che aumentino la sicurezza delle aree monitorate.

Le informazioni estratte dall'immagine presentano però distorsioni dovute alla prospettiva della camera e errori legati alla precisione dell'Object Detection.
Questa tesi descrive un possibile approccio per la correzione di queste problematiche, permettendo l'acquisizione di misure relative a posizione e velocità delle entità inquadrate.
