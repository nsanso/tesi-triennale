\chapter{Velocità}
\label{sec:velocita}

Possiamo calcolare la velocità $V$ utilizzando l'equazione \ref{eq:vel}
\begin{equation}
    \label{eq:vel}
    \vec{V}_t = \frac{(\vec{P}_t - \vec{P}_{t-1})}{\Delta}
\end{equation}
dove $\vec{P}$ è la posizione reale calcolata in precedenza e $\Delta$ è intervallo tra $t$ e $t-1$.
Calcolare la velocità in questo modo amplifica gli errori presenti nelle misure di posizione, soprattutto se $\Delta$ è piccolo.
Possiamo però sfruttare il fatto che $\Delta$ è piccolo prendendo come assunto che la differenza tra $\vec{V}_t$ e $\vec{V}_{t-1}$ sia anch'essa piccola.
Possiamo quindi provare a correggere il rumore utilizzando una media mobile esponenziale, come in \ref{eq:ewma}
\begin{equation}
    \label{eq:ewma}
    \vec{V'}_t = k \cdot \vec{V'}_{t-1} + (1-k) \cdot \vec{V}_t
\end{equation}
dove $k \in [0, 1)$ e $V$ è calcolata come in \ref{eq:vel}.
Notiamo però che quest'approccio introduce un ritardo tra i valori calcolati e quelli reali (fig \ref{fig:lag}).
È perciò necessario utilizzare un algoritmo che corregga questo ritardo.
L'algoritmo sviluppato dallo studente è espresso dalle equazioni in \ref{eq:smoothns}
\begin{equation}
    \label{eq:smoothns}
    \begin{split}
        & v_t = v'_{t-1} + v''_{t-1} \cdot \Delta \\
        & v'_t = k \cdot v_t + (1-k) \cdot V_t \\
        & v''_t = k \cdot v'_{t-1} + (1-k) \cdot (v'_t - v_t) \\
        & V'_t = v'_t + v''_t \cdot \left(\frac{1}{1-k} - 1\right) \\
    \end{split}
\end{equation}
L'algoritmo proposto da \textbf{LaViola2003} è espresso dalle equazioni in \ref{eq:laviola}
\begin{equation}
    \label{eq:laviola}
    \begin{split}
        & v'_t = \alpha \cdot V_t + (1-\alpha) \cdot v'_{t-1} \\
        & v''_t = \alpha \cdot v'_{t} + (1-\alpha) \cdot v''_{t-1} \\
        & V'_t = \left(2 + \frac{\alpha \cdot \Delta}{1 - \alpha}\right) \cdot v'_t - \left(1 + \frac{\alpha \cdot \Delta}{1 - \alpha}\right) \cdot v''_t \\
    \end{split}
\end{equation}
In fig \ref{fig:smooth} sono messi a confronto \ref{eq:vel}, \ref{eq:ewma}, \ref{eq:smoothns}, \ref{eq:laviola}.

\textbf{TODO: AGGIUNGI MOLTE IMMAGINI DI CONFRONTO, E SE IL TUO ESCE PEGGIO RIMUOVILO DA QUI E DAL PROGETTO}
