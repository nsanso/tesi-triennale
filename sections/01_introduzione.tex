\chapter{Introduzione e stato dell'arte}
\label{sec:introduzione}

% Homogenous matrix: 4 points, automatic generation; Single view metrology.
% Kalman filters, EWMA, Moving averages

\section{Contesto e motivazioni}

Lo sviluppo di un sistema di rilevazione di anomalie stradali richiede la conoscenza dei dati relativi alle entità presenti nella scena inquadrata dalla telecamera di sorveglianza.
In particolare risultano necessari i dati relativi a posizione, velocità e categoria delle entità.
La categoria è importante per differenziare persone, veicoli e oggetti, mentre posizione e velocità sono importanti per comprendere come questi si comportano e interagiscono.
Possiamo ottenere la categoria delle entità e la loro posizione nell'immagine utilizzando un Object Detector, generalmente implementato con una Convolutional Neural Network \cite{cnn}.
È necessario poi tracciare le entità tra un frame e l'altro, in modo da poterne misurare gli spostamenti e quindi la velocità.
Si ottengono buoni risultati con tracker basati su Discriminative Correlation Filters \cite{dcf}.

Sia posizione che velocità ottenute in questo modo sono in spazio immagine e non corrispondono con le misure reali.
Inoltre entrambe presentano errori simili a rumore, derivati dalla precisione non perfetta dell'Object Detection.
Per ottenere un sistema affidabile è necessario correggere in modo efficace sia la distorsione prospettica sia il rumore.

\section{Obiettivo della tesi}
\begin{itemize}
	\item Indicare una metodologia per la correzione della distorsione prospettica per inquadrature con orientamento arbitrario, da applicare al contesto stradale.
	\item Indicare un algoritmo di correzione del rumore che permetta di ricavare posizione e velocità affidabili.
\end{itemize}

\section{Attività della tesi}

\begin{itemize}
	\item Studio delle trasformazioni di proiezione
	\item Sviluppo di un algoritmo di correzione del rumore
	\item Implementazione delle tecniche scelte in un sistema di sorveglianza stradale
\end{itemize}

\section{Schema della tesi}

Nel capitolo \ref{sec:funzionalita} sono illustrati in dettaglio i problemi affrontati e il modo in cui sono affrontati allo stato dell'arte.
Nel capitolo \ref{sec:posizione} è descritta la soluzione scelta per l'acquisizione della posizione con correzione della distorsione prospettica, la sua implementazione e le problematiche riscontrate.
Nel capitolo \ref{sec:velocita} è descritta la soluzione scelta per la correzione ed l'acquisizione di misure di posizione e velocità.
Nel capitolo \ref{sec:architettura} è illustrata l'architettura del sistema di rilevazione di anomalie stradali.
Infine nel capitolo \ref{sec:conclusioni} è riassunto il lavoro svolto, sono discussi i risultati ottenuti e sono suggeriti possibili sviluppi e miglioramenti.
