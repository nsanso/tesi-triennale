\chapter{Introduzione e stato dell'arte}
\label{sec:introduzione}

La correzione della \emph{distorsione prospettica} è un passaggio necessario nella maggioranza dei processi che estraggono informazioni da immagini acquisite attraverso una camera.
Nei casi in cui tale distorsione sia poco accentuata è possibile approssimare una correzione utilizzando trasformazioni lineari come rotazione, scalatura e shear dell'immagine.
% Negli altri casi risulta necessario un approccio più profondo, che richiede un cambio di coordinate da cartesiane a proiettive omogenee e la generazione di una matrice omografica che trasformi lo \emph{spazio immagine} in \emph{spazio reale}.
% Spesso per semplicità si riduce lo \emph{spazio reale} ad un piano approssimabile alla superficie dell'oggetto di interesse.
Negli altri casi risulta necessario un approccio più profondo: si limita lo \emph{spazio reale} preso in considerazione a un piano bidimensionale e si individua la trasformazione che proietta lo \emph{spazio immagine} nel piano scelto.
Tale trasformazione è rappresentata attraverso una matrice omografica in coordinate omogenee.
Lo studio di come generare la matrice di trasformazione in questione è esplorato in modo particolarmente esteso nel campo del processing di documenti; i documenti presentano caratteristiche facilemente individuabili su cui poter fare leva per ricavare informazioni sulla distorsione dell'immagine.
È infatti possibile sfruttare le linee parallele e perpendicolari nei bordi e nel testo per ricavare la matrice di trasformazione desiderata \cite{persp:docs}.
Un possibile approccio nel campo della video sorveglianza è quello di sfruttare un dispositivo di acquisizione immagini in grado di identificare la profondità di ogni pixel dell'immagine inquadrata e con questa ricavare la trasformazione voluta \cite{persp:depth}.
Un'ulteriore alternativa è quella di individuare una corrispondenza tra 4 punti conosciuti nello \emph{spazio reale} e nello \emph{spazio immagine} e utilizzare tale corrispondenza per generare la matrice omografica \cite{persp:map}.
% Quest'ultimo metodo è spesso utilizzato anche quando non si conoscono esattamente le coordinate dei punti nello \emph{spazio reale}, ma si conosce la loro disposizione. Individuando 4 punti che nello \emph{spazio reale} delineano un quadrato e correlandoli con i corrispettivi 4 punti nello \emph{spazio immagine} è possibile generare facilmente una matrice di correzione. Tale matrice non garantisce però magnitudini corrette e corregge solamente angoli e rapporto tra gli assi.

Nonostante tutti i metodi sopra siano ottime soluzioni, nessuna di queste è facilmente applicabile senza enormi sforzi: il metodo della mappatura richiede la conoscenza di misurazioni precise per ogni camera implementata nel sistema; il metodo della derivazione dalla profondità richiede la spesa per l'acquisto di camere in grado di rilevarla; i metodi che utilizzano linee parallele e perpendicolari, utilizzati per i documenti, sono probabilmente i più interessanti ma richiedono un enorme sforzo implementativo per essere applicati al nostro contesto.

\section{Contesto e motivazioni}

Lo sviluppo di un sistema di rilevazione di anomalie stradali richiede la conoscenza dei dati relativi alle entità presenti nella scena inquadrata dalla telecamera di sorveglianza.
In particolare risultano necessari i dati relativi a posizione, velocità e categoria delle entità.
La posizione e la categoria possono essere ricavate utilizzando una \emph{Convolutional Neural Network (CNN)} che esegue l'\emph{Object Detection}.
La velocità può poi essere ricavata applicando un algoritmo di tracciamento ai dati ottenuti nella fase di categorizzazione.

Questi dati sono però in \emph{spazio immagine} e non in \emph{spazio reale}.
Questi due spazi differiscono in numero di dimensioni, essendo lo \emph{spazio immagine} bidimensionale e lo \emph{spazio reale} tridimensionale.
Non esiste perciò una trasformazione lineare dall'uno all'altro.
Utilizzare le misurazioni in \emph{spazio immagine} per la rilevazione di anomalie porta a un sistema fragile e poco affidabile, in quanto tali misurazioni risultano distorte rispetto a i loro valori reali.

È perciò necessario individuare una trasformazione da \emph{spazio immagine} a \emph{spazio reale}.
Deve essere possibile arrivare a tale trasformazione senza conoscere i dati reali relativi alla camera e alla superficie inquadrata, in quanto questi non sono sempre disponibili.
Inoltre applicare questa trasformazione deve necessariamente essere poco costoso computazionalmente, in modo da mantenere il sistema di rilevazione di anomalie eseguibile in real time.

\section{Obiettivo della tesi}
\begin{itemize}
	\item Sviluppare un algoritmo di correzione prospettica applicabile ad un sistema di rilevazione di anomalie nel contesto della sicurezza stradale.
	\item Sviluppare e implementare un tool interattivo per la generazione di dati relativi alla distorsione prospettica della singola camera.
\end{itemize}

\section{Attività della tesi}

\begin{itemize}
	\item Studio delle trasformazioni di proiezione
	\item Sviluppo del tool interattivo per la generazione delle informazioni relative alla camera
	\item Implementazione del modulo di correzione nel sistema real time di rilevazione anomalie
	\item Confronto tra risultati con e senza correzione
\end{itemize}

\section{Schema della tesi}

Nel capitolo \ref{sec:teoria} è illustrato il problema ed è modellato dal punto di vista matematico. In questo capitolo vengono discusse e giustificate le formule applicate.

Nel capitolo \ref{sec:funzionalita} sono illustrate le funzionalità richieste ai due sistemi, quello di correzione e quello di generazione della trasformazione correttiva, e sono discusse le scelte compiute per ottenerle.

Nel capitolo \ref{sec:implementazione} è descritta brevemente l'architettura del sistema con cui il modulo di correzione interagisce e sono discusse le tecnologie e le metodologie implementative utilizzate.

Nel capitolo \ref{sec:testing} è descritto il processo di validazione e valutazione della soluzione proposta e i suoi risultati.

Infine nel capitolo \ref{sec:conclusioni} è riassunto il lavoro svolto, sono discussi i risultati ottenuti e sono suggeriti possibili sviluppi e miglioramenti,
