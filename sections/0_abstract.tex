\chapter*{Abstract}
Data l'alta diffusione di sistemi di video sorveglianza a camera fissa (CCTV) risulta interessante lo sviluppo di sistemi in grado di individuare, tracciare e monitorare le entità inquadrate da tale camera.
Le misure estratte dall'immagine, come posizione e velocità, risultano però distorte rispetto alla realtà. 
Questa distorsione è legata alla posizione e alla rotazione della camera rispetto alla scena inquadrata.
Chiamiamo questa distorsione \emph{distorsione prospettica}.
Lo scopo di questa tesi è quello di presentare un algoritmo di correzione della \emph{distorsione prospettica} e una metodologia interattiva per la generazione delle informazioni necessarie a tale algoritmo.
