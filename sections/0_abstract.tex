\chapter*{Abstract}
Data l'alta diffusione di sistemi di video sorveglianza a camera fissa (CCTV) risulta interessante lo sviluppo di sistemi in grado di individuare, tracciare e monitorare le entità inquadrate da tale camera.
Le misure, come posizione e velocità, estratte dall'immagine risultano però distorte rispetto alla realtà. Questa distorsione è legata a posizione e rotazione della camera e della superficie inquadrata e non è lineare.
Chiamiamo questa distorsione \emph{distorsione prospettica}.
Lo scopo di questa tesi è quello di presentare un algoritmo di correzione della \emph{distorsione prospettica} e una metodologia interattiva per la generazione delle informazioni necessarie a tale algoritmo.
