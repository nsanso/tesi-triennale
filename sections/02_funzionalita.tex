\chapter{Funzionalità}
\label{sec:funzionalita}

Il sistema da noi sviluppato per monitorare una scena urbana per la rilevazione di anomalie deve essere in grado di adempiere a vari compiti:
\begin{enumerate}
    \item \label{itm:1} Rilevare le entità presenti nella scena
    \item \label{itm:2} Tracciare le entità nel tempo, associando loro un ID
    \item \label{itm:3} Misurare posizione e velocità delle entità tracciate
    \item \label{itm:4} Rilevare, in base alle misure effettuate, quando si verifica un'anomalia e da quali entità essa è causata
    \item \label{itm:5} Notificare un sistema esterno dell'avvenuta anomalia
\end{enumerate}

L'utilizzo di un Object Detector (e.g. Yolo \cite{arch:yolo}) permette di completare il punto \ref{itm:1}.
Il punto \ref{itm:2} è gestito utilizzando algoritmi di tracking, come quelli basati su DCF \cite{dcf}.
Il punto \ref{itm:4} è svolto utilizzando euristiche sviluppate e validate sperimentalmente che sfuttano le misure ottenute nel punto \ref{itm:3}. 
Nel punto \ref{itm:5} utilizziamo il protocollo HTTP per inoltrare le anomalie ad un microservizio attivo sullo stesso dispositivo, che gestisce la comunicazione con sistemi esterni.
Il punto \ref{itm:3} richiede la correzione della distorsione prospettica e del rumore presente nelle misurazioni per essere utile alle euristiche del punto \ref{itm:4}, ed è l'argomento principale di questa tesi.

\section{Posizione}
\label{sec:funzionalita-posizione}
Per poter misurare distanze e velocità delle entità è necessario conoscerne la posizione reale.
La posizione estratta dall'immagine non corrisponde però a quella reale, in quanto presenta una distorsione prospettica legata alle carattestiche e posizione della camera da cui è acquisita.
La correzione della distorsione prospettica si divide in due fasi:
\begin{enumerate}
    \item Comprensione della geometria tridimensionale della scena rappresentata nell'immagine
    \item Proiezione dell'immagine sulla scena ricostruita.
\end{enumerate}
Processi che eseguono entrambe le fasi in modo completo ed accurato restituiscono una approssimazione tridimensionale della scena, e utilizzano informazioni di profondità ottenute o direttamente \cite{persp:depth}, utilizzando per esempio camere stereo, o estratte dalle caratteristiche dell'immagine \cite{svm}.
Dovendo però ricostruire una scena di contesto stradale possiamo assumere che il manto stradale corrisponda ad un piano, e limitare le nostre operazioni in questo nuovo dominio.
Questo ci permette di ridurre il problema ad individuazione e applicazione di una matrice di proiezione prospettica dal piano immagine al piano stradale.
La soluzione più usata è quella di calcolare direttamente la matrice mettendo in corrispondenza 4 punti nell'immagine con 4 punti nel piano reale \cite{persp:map}.
Questo richiede però conoscere esattamente la posizione dei 4 punti, e quindi effettuare misure sul campo.
Non è quindi una soluzione accettabile per il nostro sistema, in quanto vogliamo essere in grado di aggiungere nuove inquadrature lavorando esclusivamente in remoto.
Questa tesi studia la costruzione della matrice di proiezione prospettica per individuare una soluzione adatta al nostro sistema.

\section{Velocità}
\label{sec:funzionalita-velocita}
Possiamo ricavare la velocità reale delle entità utilizzando le posizioni corrette in precedenza.
La velocità così misurata risulta caotica, in quanto gli errori di precisione presenti nelle misure di posizione vengono amplificati significativamente.
È quindi necessario utilizzare un algoritmo di smoothing per correggere questo rumore e rendere utili le nostre misure.
La soluzione più utilizzata allo stato dell'arte per la soluzione di queste problematiche è il Kalman Filter e le sue estensioni \cite{kalman}, ma questo risulta difficile da comprendere e complesso da implementare.
È stato dimostrato da \cite{laviola} che è possibile ottenere risultati simili a quelli ottenuti con il Kalman Filter utilizzando algoritmi basati sulla media mobile esponenziale, che risultano più semplici e veloci da implementare.
È importante notare che queste soluzioni ottengono risultati simili al Kalman Filter solamente in alcune situazioni, in particolare quelle in cui la fiducia nelle nostre misure è alta.
È quindi preferibile utilizzare il Kalman Filter se il costo di implementazione è sostenibile.
Noi abbiamo sviluppato un algoritmo alternativo, che non siamo riusciti a trovare in letteratura, che permette di correggere e simultaneamente calcolare posizione e velocità in modo efficiente.
