\chapter{Funzionalità}
\label{sec:funzionalita}

Il sistema da noi sviluppato per monitorare una scena urbana per la rilevazione di anomalie deve essere in grado di adempiere a vari compiti:
\begin{enumerate}
    \item \label{itm:1} Rilevare le entità presenti nella scena
    \item \label{itm:2} Tracciare le entità nel tempo, associando loro un ID
    \item \label{itm:3} Misurare posizione e velocità delle entità tracciate
    \item \label{itm:4} Rilevare, in base alle misure effettuate, quando si verifica un'anomalia e da quali entità essa è causata
    \item \label{itm:5} Notificare un sistema esterno dell'avvenuta anomalia
\end{enumerate}

L'utilizzo di un Object Detector come Yolo \textbf{CITAZIONE} permette di completare il punto \ref{itm:1}.
Il punto \ref{itm:2} è gestito utilizzando algoritmi di tracking \textbf{TRACKER NVIDIA}. In questa fase sono inoltre utilizzati algoritmi di smoothing per ridurre l'errore presente nelle posizioni ottenute al punto \ref{itm:1}.
Il punto \ref{itm:4} è svolto utilizzando euristiche sviluppate e validate sperimentalmente. 
Il punto \ref{itm:5} utilizza il protocollo HTTP per inoltrare le anomalie ad un microservizio attivo sullo stesso dispositivo, che gestisce la comunicazione con sistemi esterni.
Il punto \ref{itm:3} richiede la correzione della distorsione prospettica e del rumore presente nelle misurazioni per essere utile alle euristiche del punto \ref{itm:4}, ed è l'argomento principale di questa tesi.

\section{Distorsione prospettica}
\label{sec:funzionalita-prospettiva}
La correzione della distorsione prospettica si divide in due fasi:
\begin{enumerate}
    \item Comprensione della geometria tridimensionale della scena rappresentata nell'immagine
    \item Proiezione dell'immagine sulla scena ricostruita.
\end{enumerate}
Processi che eseguono entrambe le fasi in modo completo ed accurato restituiscono una approssimazione tridimensionale della scena, e utilizzano informazioni di profondità ottenute o direttamente \textbf{CITAZIONE}, utilizzando per esempio camere stereo, o estratte dalle caratteristiche dell'immagine \textbf{CITAZIONE}.
Dovendo però ricostruire una scena di contesto stradale possiamo assumere che il manto stradale corrisponda ad un piano, e limitare le nostre operazioni in questo nuovo dominio.
Questo ci permette di ridurre il problema ad individuazione e applicazione di una matrice di proiezione prospettica dal piano immagine al piano stradale.
La soluzione più usata è quella di calcolare direttamente la matrice mettendo in corrispondenza 4 punti nell'immagine con 4 punti nel piano reale \textbf{CITAZIONE}.
Questo richiede però conoscere esattamente la posizione dei 4 punti, e quindi effettuare misure sul campo.
Non è quindi una soluzione accettabile per il nostro sistema, in quanto vogliamo essere in grado di aggiungere nuove inquadrature lavorando esclusivamente in remoto.
Questa tesi studia la costruzione della matrice di proiezione prospettica per individuare una soluzione adatta al nostro sistema.
