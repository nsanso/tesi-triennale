\chapter{Funzionalità}
\label{sec:funzionalita}

Il sistema da noi sviluppato per monitorare una scena urbana per la rilevazione di anomalie deve essere in grado di adempiere a vari compiti:
\begin{enumerate}
    \item \label{itm:1} Rilevare le entità presenti nella scena
    \item \label{itm:2} Tracciare le entità nel tempo, associando loro un ID
    \item \label{itm:3} Misurare posizione e velocità delle entità tracciate
    \item \label{itm:4} Rilevare, in base alle misure effettuate, quando si verifica un'anomalia e da quali entità essa è causata
    \item \label{itm:5} Notificare un sistema esterno dell'avvenuta anomalia
\end{enumerate}

L'utilizzo di un Object Detector come Yolo \textbf{TODO: VERSIONE SPECIFICA E REF} permette di completare il punto \ref{itm:1}.
Il punto \ref{itm:2} può essere svolto utilizzando le misure ottenute nel punto \ref{itm:3} e predicendo la posizione delle entità nei frame successivi. 
Questa predizione può essere utilizzata per eseguire il match tra le entità già conosciute e quelle rilevate nel frame corrente. 
Utilizzando un tracker capace di \textbf{TODO: SUGGERISCI UN TRACKER COME QUELLO DI NVIDIA} si ottengono risultati migliori di quelli ottenuti eseguendo il match utilizzando solamente posizione, velocità, categoria e caratteristiche del bounding box dell'entità.
Il punto \ref{itm:4} è svolto utilizzando euristiche sviluppate e validate sperimentalmente. 
Il punto \ref{itm:5} utilizza il protocollo HTTP per inoltrare le anomalie ad un microservizio attivo sullo stesso dispositivo, che gestisce la comunicazione con sistemi esterni.
Il punto \ref{itm:3} richiede la correzione della distorsione prospettica e del rumore presente nelle misurazioni per essere utile alle euristiche del punto \ref{itm:4}, ed è l'argomento principale di questa tesi.

\section{Distorsione prospettica}
\label{sec:funzionalita-prospettiva}
La correzione della distorsione prospettica si divide in due fasi:
\begin{enumerate}
    \item Comprensione della geometria tridimensionale della scena rappresentata nell'immagine
    \item Proiezione dell'immagine sulla scena ricostruita.
\end{enumerate}
Processi che eseguono entrambe le fasi in modo completo ed accurato restituiscono una approssimazione tridimensionale della scena, e utilizzano informazioni di profondità ottenute o direttamente, utilizzando per esempio camere stereo, o estratte dalle caratteristiche dell'immagine.

Dovendo però ricostruire una scena di contesto stradale possiamo assumere che il manto stradale corrisponda ad un piano, e limitare le nostre operazioni in questo nuovo dominio.
Questo ci permette di ridurre il problema ad individuazione e applicazione di una matrice di proiezione prospettica dal piano immagine al piano stradale.

\textbf{TODO: blah blah blah Documenti e OCR, Quattro punti, Non danno misure precise lo faccio interattivo}


\section{Correzione del rumore}
\label{sec:funzionalita-rumore}
Le \emph{CNN} per l'\emph{Object Detection} che utilizziamo per rilevare le entità all'interno dello stream video trattano ogni frame indipendentemente dagli altri.
Per questo motivo, e per il fatto che il processo di individuazione è un processo probabilistico, i Bounding Boxes ottenuti non avvolgono perfettamente le entità.
Le imprecisioni presenti in tali Bounding Boxes non presentano bias e raramente si allontanano in modo significativo dalla loro posizione ideale.
Per questi motivi possiamo trattare le imprecisioni come rumore con distribuzione normale, e correggerlo quindi con operazioni di media continua.
Stiamo però cercando di ottenere la posizione e la velocità di entità in movimento, ed entrambe le variabili misurate possono cambiare il loro valore nel periodo in cui l'entità è tracciata.

\textbf{DA QUI IN POI VA RISCRITTO CON PIÙ CITAZIONI E MENO A CASO} \\
Tale problema è trattato spesso in contesti di tracking, e la soluzione allo stato dell'arte è quella di utilizzare Kalman Filters.
Questi permettono di aggiornare più variabili in base a conoscenze a priori e a nuove misure in modo pesato.
Più siamo sicuri delle nostre conoscenze e più gli diamo peso, ma se queste si discostano in modo significativo dalle nuove misure, o la nostra sicurezza diminuisce, queste nuove misure diventano più significative anche nei nostri calcoli.
È possibile simulare questo comportamento anche utilizzando altri algoritmi di correzione che sfruttano l'EWMA (Exponentially weighted moving average) e questa è la soluzione descritta in questa tesi.
