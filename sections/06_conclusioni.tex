\chapter{Conclusioni}
\label{sec:conclusioni}

L'obiettivo iniziale di correggere la \emph{distorsione prospettica} presente nelle misure di posizione e velocità è stato perciò raggiunto.
Questo risultato è stato ottenuto modellando il problema matematicamente in modo da comprendere la natura della distorsione e da individuare le approssimazioni necessarie alla sua risoluzione.
La fase complicata si è rivelata essere la generazione della trasformazione correttiva e non la sua applicazione.
È stato perciò necessario sviluppare uno strumento esterno interattivo con cui semplificare tale generazione.
Il lavoro fatto è stato poi validato confrontando i risultati ottenuti dal sistema di rilevazione di anomalie prima e dopo l'implementazione della soluzione proposta.

Possibili sviluppi futuri della soluzione trovata sono il permettere la definizione di trasformazioni sulla stessa immagine, in modo da gestire ambienti in cui un singolo piano non rende un'approssimazione soddisfacente.
Inoltre potrebbe essere interessante implementare i metodi discussi nella sezione riguardante lo stato dell'arte (Capitolo \ref{sec:introduzione}) nello strumento interattivo, come supporto all'operatore.

% NON SUDDIVISA per punti ma discorsiva (io suddivido per punti per chiarezza):
% •	l'obiettivo era (notare il tempo al passato) bl bla ripeti quanto detto in cap 1
% •	Abbiamo perseguito gli scopi preposti facendo bla bla ribadisci quanto raccontato nella tesi
% •	I risultati sono bla bla (sottolinea le cose + salienti
% •	commenti, limiti del lavoro, considerazioni, conclusioni e sviluppi futuri.
