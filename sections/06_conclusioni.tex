\chapter{Conclusioni}
\label{sec:conclusioni}

Abbiamo descritto un possibile approccio che permette di ricavare informazioni relative a posizione e velocità delle entità presenti nell'inquadratura di un sistema di sorveglianza stradale.
L'approccio descritto richiede che siano verificate varie condizioni, come per esempio che un piano sia una buona approssimazione del manto stradale e che l'intervallo di tempo tra una misura e l'altra sia piccolo e costante.
Nel caso questi non si verifichino nella nostra scena è possibile utilizzare soluzioni più sofisticate, come per esempio la Single View Metrology \cite{svm} per i problemi di prospettiva e Kalman Filter ed estensioni \cite{kalman} per i problemi di precisione nelle misure.

In questa tesi abbiamo lasciato alcune problematice aperte per quanto riguarda il posizionamento delle entità:
\begin{itemize}
    \item a seconda della direzione di movimento e della dimensione delle entità, l'approssimazione della posizione può essere più o meno buona.
    \item non è trattato un metodo per approssimare la forma delle entità, necessaria per lo sviluppo di un sistema in grado di rilevare urti tra di esse.
\end{itemize}
Queste problematiche rimangono al momento irrisolte.
