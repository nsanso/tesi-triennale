\chapter{Architettura}
\label{sec:architettura}

% TODO: AGGIUNGI IMMAGINI, RIFERIMENTI 

% Definire e desccrivere nei dettagli l'architettura a blocchi, descrivendo ciascun blocco e i collegamenti.
% Non è ancora necessario dare dettagli implementativi, che andranno nel capitolo dedicato.
% Descrizione tecnologie utilizzate per implementare ciascun blocco.
% Qui vanno i dettagli implementativi per ciascun blocco dell'architettura definita nel capitolo precedente.
% Potrebbe convenire replicare l'immagine dell'architettura, definendo all'interno dell'immagine dettagli implementativi come ad esempio i nomi dei moduli utilizzati e le tecnologie utilizzate per implementarli.

\textbf{TODO: AGGIUNGI IMMAGINE}

L'architettura scelta è una pipeline, in quanto l'input del sistema è uno stream video.
Ogni frame dello stream deve essere processato nello stesso modo, e utilizzare una pipeline consente di aggiungere, rimuovere o sostituire moduli, e quindi modificare il processo, in modo semplice e veloce.

Il sistema è implementato sul dispositivo \emph{Nvidia Jetson Xaxier}\cite{arch:jetson} ed è sviluppato su \emph{Ubuntu 18.04}\cite{arch:ubuntu}.
Il linguaggio di sviluppo è \emph{Python 3.6}\cite{arch:python}.
Per la gestione delle operazione di algebra lineare sono stati utilizzati i package \emph{Numpy}\cite{arch:numpy} e \emph{SciPy}\cite{arch:scipy}.
La pipeline di acquisizione è gestita dalla libreria \emph{GStreamer}\cite{arch:gstreamer}.
La classificazione è effettuata dal modulo di inferenza fornito dall'\emph{SDK Nvidia Deepstream}\cite{arch:deepstream}.
Questo modulo di inferenza è configurato per il modello di \emph{Object Detection Scaled-YoloV4}\cite{arch:yolo}.
